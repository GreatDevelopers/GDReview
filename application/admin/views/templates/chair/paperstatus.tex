% Use your own class of Latex document here
%
% IMPORTANT: since data comes from the Web, it may contains weird
% characters. The simplest solution is to compile with xelatex. 
%

\documentclass{article}
\usepackage[T1]{fontenc}
\usepackage{multicol}
\usepackage[english]{babel}
\usepackage[final]{pdfpages}
\usepackage[margin=2cm,includefoot,includehead]{geometry}
\usepackage{makeidx}
\usepackage{index}
\usepackage[bookmarks=yes]{hyperref}

\makeindex

% Declare the index of authors. Note: the index must
% be produced with the following command (after a first Latex compilation):
% makeindex proceedings.ax -o proceedings.ad
%\newindex{authors}{ax}{ad}{Index of authors}

% Some pdfpages parameters
%\includepdfset{pages=-,pagecommand={}}

% OK, here begins the document
\begin{document}

% The title page
\title{ \textbf{Status of submissions to {Config->confName}}}
\author{ {Config->chair_names} }
\date{\today}

\maketitle

% The table of contents
\tableofcontents

%\part{Papers ranked on their overall marks, in desc. order}

    <!-- BEGIN REVIEW_CRITERIA -->
    
     <!-- END REVIEW_CRITERIA -->


<!-- BEGIN PAPER_DETAIL -->

 <!-- BEGIN PAPER_INFO -->

\section{Submission({paper_id}) {paper_title} -- Overall: {paper_overall}}  

\textit{Authors}: {paper_authors}
 <!-- END PAPER_INFO -->

     <!-- BEGIN REVIEWER -->
 \subsection{Reviewer:  {reviewer_fname} {reviewer_lname}}
    
   Level of expertise for this submission: {reviewer_expertise}.   
     <!-- END REVIEWER -->
\begin{enumerate}
    <!-- BEGIN REVIEW_MARK -->
      \item \textbf{{criteria_label}}: {ReviewMark->mark}
    <!-- END REVIEW_MARK -->
 \end{enumerate}
External reviewer:   {external_reviewer_fname}  {external_reviewer_lname} 
   \subsubsection*{Summary}
   {review_summary}
      
   \subsubsection*{Details}
   {review_details}

   \subsubsection*{Comments for PC}
   {review_comments}
<!-- END PAPER_DETAIL -->

% Print the index of authors
%\addcontentsline{toc}{chapter}{Index of authors}
%\printindex[authors]

\end{document}